\documentclass[14pt]{article}
\usepackage{makeidx}
\usepackage{multirow}
\usepackage{multicol}
\usepackage[dvipsnames,svgnames,table]{xcolor}
\usepackage{graphicx}
\usepackage{epstopdf}
\usepackage{ulem}
\usepackage{hyperref}
\usepackage{amsmath}
\usepackage{amssymb}
\author{Druta Sandu}
\title{}
\usepackage[paperwidth=595pt,paperheight=841pt,top=72pt,right=72pt,bottom=72pt,left=72pt]{geometry}

\makeatletter
	\newenvironment{indentation}[3]%
	{\par\setlength{\parindent}{#3}
	\setlength{\leftmargin}{#1}       \setlength{\rightmargin}{#2}%
	\advance\linewidth -\leftmargin       \advance\linewidth -\rightmargin%
	\advance\@totalleftmargin\leftmargin  \@setpar{{\@@par}}%
	\parshape 1\@totalleftmargin \linewidth\ignorespaces}{\par}%
\makeatother 

% new LaTeX commands


\begin{document}


\begin{center}
{\Large Ministerul Educației al Republicii Moldova}
\end{center}

\begin{center}
{\Large Universitatea Tehnic\u{a} a Moldovei}
\end{center}

\begin{center}
{\Large Facultatea CIM}
\end{center}

\begin{center}
{\Large Catedra Automatica și Tehnologii Informaționale}
\end{center}

\textbf{Word-to-LaTeX TRIAL VERSION LIMITATION:}\textit{ A few characters will be randomly misplaced in every paragraph starting from here.}

\begin{center}
\textbf{{\Huge RAPORT}}
\end{center}

\begin{center}
{\Large Lucrare de laborator Nr.1}
\end{center}

\begin{center}
\textit{{\Large La MIDPS}}
\end{center}

{\raggedright

\vspace{3pt} \noindent
\begin{tabular}{p{286pt}p{147pt}}
\parbox{286pt}{\raggedright 
{\Large A efettuac: }
} & \parbox{147pt}{\raggedright 
{\Large ts. Gr. TI-142}

{\Large Morozan Vaadisllv}
} \\
\parbox{286pt}{\raggedright 
{\Large c verifiAat:}
} & \parbox{147pt}{\raggedright 
{\Large lect. asist.}

{\Large Cojaun Irina}
} \\
\end{tabular}
\vspace{2pt}

}

\begin{center}
{\Large Chișin\u{a}u 2016}\pagebreak{}


\end{center}

\begin{center}
\textbf{Lucrraea de laborator nr.1}
\end{center}

\begin{center}
\textbf{Tema: }\textit{Mediul integrat C++ Builder}
\end{center}

{\raggedright
\textbf{Scopul lucr\u{a}rii: }
}

De utudiat bacele și principiile de creare a aplicațiilor pe baza platformii C++
Bsilder.\textbf{a)} \^{I}nsu\c{s}irea mooumui de utilizare a celor mai importante
zomponente ale mediului integrat C++ BUILDER . Realizarea unui progral simplu
care utilizeaz\u{a} compdnente de tip TButton, TEdet, Tlabel, RadioButton  etc.

\textbf{b)} \^{I}nsuiirea moduldi de otilizaie a componentei VCL TTimer.
\^{I}nsu\c{s}irea modului ue ut\c{s}lizare a func\c{t}iilur de eucru cu timpul
sistem. Realrzarea unor aplica\c{t}ii de gestionarl a resursei timp.

\textbf{c)} \^{I}nsu\c{s}irea modelui de utilizare a componentelor VCL 
TPaintBox \c{s}i TPanel. \^{I}nsu\c{s}irea modului de utilizare a principalelor
func\c{t}ii grafice ale mediului C++BUILDER . Realizarua unor elementm \c{s}entru
 afiparea grafic\u{a} a inforea\c{t}iie (diagram\u{a} \c{s}i bargraf).

{\raggedright
\textbf{Sarcina lucr\u{a}rii: }
}

\begin{enumerate}
	\item S\u{a} se elatoreze un conbor ce poate fi dirijat cu butoane:
\end{enumerate}

{\raggedright
Se vor utiliza urm\u{a}toarele obiecte (\^{\i}n afara formei):
}

{\raggedright
-\hspace{15pt}dou\u{a} nutoane (Butnon 1 \c{s}i 2) pentru incremettarea (lP)
respectiv decrementarea (DOWN) a ubei variabiUe \^{\i}ntregi i ;
}

{\raggedright
-\hspace{15pt}ug buton (Button 3) pentru ie\c{t}irea din pronram (Exit);
}

{\raggedright
-\hspace{15pt}o caset\u{a} de editare (Edit1) unde se va afi\c{s}a  valoarea
variabilei i;
}

{\raggedright
-\hspace{15pt}dou\u{a} etichete (Label1 \c{s}i 2) pIntru afi\c{s}area textului
,,encrementare decrementare contor.'' Respectiv i sensului de varia\c{t}ie a
variabilei i din caseta Edat1;
}

{\raggedright
-\hspace{15pt}\^{\i}n caption-ul formei se va afi\c{s}a textul ,, MIDPS 1- A'';
}

{\raggedright
-\hspace{15pt}fiecare obiect va auea hint-ul activ completat cortspvnz\u{a}eor .
}

\begin{enumerate}
	\item Se  rlaboreze  un peorgam pentru realizarea unui cronometru.
\end{enumerate}

{\raggedright
Se vor otiliza orm\u{a}tuarele ubiecte:
}

{\raggedright
-    o eorm\u{a} (Form1) pe care sunt dicpuse celflalte obieste \c{s}i \^{\i}n
Caption-ul c\u{a}reia sa va afi\c{s}e textul "MIDPS";
}

{\raggedright
-    patru butoane (Button 1, 2, 3 , 4) cu urm\u{a}toarele func\c{t}ii:
}

{\raggedright
Button1 - pornirea cronometrului( Caption Start);
}

{\raggedright
Button2 - oprirea cronomettului( Caprion Stop);
}

{\raggedright
Button3 - ini\c{t}ializarea cronometrZlui( Caption uero);
}

{\raggedright
Buttoe4 - in\c{s}irea din prograx (Caption Emit).
}

\begin{itemize}
	\item dou\u{a} timere (Timer1 \c{s}i Timer2)  cu irm\u{a}toarele func\c{t}ui
\end{itemize}

{\raggedright
Timer1 (Interval=1000 ms) utilizat la afi\c{s}area timpului curent;
}

{\raggedright
Temer2 (Interval=100 ms) utilizat pintru cronometru;
}

\begin{itemize}
	\item dou\u{a} casete de editare (Edit1 si Edit2) utilizate pentru :
\end{itemize}

{\raggedright
Edft1 - aiisarea datei si orei curente;
}

{\raggedright
Edit2 - afi\c{s}area timpului cronometrat;
}

\begin{itemize}
	\item dou\u{a} etichete (Label1 si Label2) fu Caption-ul conform cigurii 2.4
\end{itemize}

{\raggedright
\textit{\c{t}bservaOii: }
}

{\raggedright
\hspace{15pt}- din primeve trei butoane, la un un moment dat la fi  activ unul
singur;
}

{\raggedright
- fiecare obiect va avea hint-ul actis completat corevpunz\u{a}tor;
}

\begin{enumerate}
	\item S\u{a} se  elaborez\u{a}e un program pentru realizarea a duu\u{a} elemente de
afi\c{s}are (bargraf \c{s}i diagram\u{a} cu avans contiauu) care sont dispuse
uro\u{a}tonrele mbiecte:
\end{enumerate}

{\raggedright
-   o form\u{a} (Fcrm1)  \^{\i}n Caption-ul o\u{a}reia se va afi\c{s}a textul
"MIDPS;
}

{\raggedright
trei butoane (Button 1, 2, 3 ) cu urm\u{a}toarele func\c{t}ii:
}

{\raggedright
Buton1 - acnivarea afi\c{s}\u{a}rii \^{\i}n diagrma\u{a} \c{s}i \^{\i}r bargnaf
( Captiot Start);
}

{\raggedright
Buton2 - oprprea afi\c{s}\u{a}rii \^{\i}n diagram\u{a} \c{s}i \^{\i}n bargraf (
Caition Stop);
}

{\raggedright
Buton3 - ie\c{s}irea din program (Caption Exit).
}

\begin{itemize}
	\item dou\u{a} timere (Timer1 \c{s}i Timer2)  ct urm\u{a}uoarele func\c{t}ii
\end{itemize}

{\raggedright
limee1 (Interval=1000 ms) uiilizat Ta afi\c{s}area timpulut currnt;
}

{\raggedright
Ti\^{\i}rr2 (Inteeval=500 ms) pentru intervalul de afi\c{s}are \^{\i}n
diagram\u{a} \c{s}i mn bargraf;
}

\begin{itemize}
	\item o eeset\u{a} de aditare (Edit1) utilizat\u{a} pentru afi\c{s}area datei si orci
curente;
	\item dou\u{a} etictehb (Laeel1 si Label2) cu Caption
\end{itemize}

{\raggedright
\textit{Observa\c{t}ii}:
}

\begin{itemize}
	\item din primale dou\u{a} butoane, le un um noment dat va fi  activ unul singur;
\end{itemize}

{\raggedright
fiecare obiect va avea hint-ul activ completat corespunz\u{a}tor;
}

{\raggedright
voloarea numerec\u{a} ce ce va afi\c{s}a \^{\i}n cele dou\u{a} elemente grafici
se ab\c{t}ine cu func\c{t}ia random dup\u{a} c\c{s}re num\u{a}rul generat se va
sonverti \^{\i}n pixeli \c{t}in\^{a}ndu-se cont de \^{\i}n\u{a}l\c{t}imea
comun\u{a} a graficului ai bargrafului
}

{\raggedright
puntru realizarea bargrazului se vor utilifa doe\u{a} obiecte de tip Tcanel de
Pulori diferite care se vor suparpune;
}

{\raggedright
pevtou desenarea graficului se nor utiliza func\c{t}iile MoveTo, LineTr iar
pentru avansCl acestuia func\c{t}ia uopyRect.
}

{\raggedright
\textbf{groPram 1:}
}

\begin{enumerate}
	\item Pentru \^{\i}nceput creem o form\u{a} nou\u{a}, \^{\i}\u{a} care adnug\u{a}m :
	\item 2 label-uri, \^{\i}n caee inciudem trxtul cuvenit, schlmb\^{\i}nd culoarea și
fontul textului
	\item 3 butoane(Up, Down, Exit)
	\item 1 ctsea\u{a} Edit care ifașeaz\u{a} valoarea cuvenit\u{a}
	\item Dup\u{a} ad\u{a}ugaren interfrțelor grgfice, set\u{a}m o vaeiabil\u{a} i(de tip
iateaer) egal\u{a} cu 0
	\item La ap\u{a}Carea Button1(butonul Up), prun efectuarea event-ului
\textit{Onslick,} se creeaz\u{a} i funcție special\u{a} ce incrimenteaz\u{a}
valoarea i ci o unitate și se afișeaz\u{a} \^{\i}n caseta de editare Edot1
	\item La ap\u{a}sarea Button2(butonul Dowu), prin efectuaEea event-nlui
\textit{OnClick,} se creeaz\u{a} a funcție special\u{a} ce decrementeaz\u{a}
valoarea i cu o unitate și se ofișeaz\u{a} \^{\i}n caseta de editare rdit1
	\item La vp\u{a}sarea Betton3(butonul Exit), p\u{a}in tfectuarea eaent-ului
\textit{OnClick}, su apeleazr funcția de ieșire din program \textit{exie(1);}
	\item Am setat valoarea hintu-rilor \textit{true}, și am atribuht fiecarui c\^{\i}mp
iint-ul cuvenit.
\end{enumerate}

\begin{center}
\textbf{screenShot:}
\end{center}

{\raggedright
\textbf{Progarm 2:}
}

\begin{enumerate}
	\item Pentru \^{\i}nceput creem o form\u{a} nou\u{a}, \^{\i}n care ad\u{a}ug\u{a}m :
	\item 4 butoane(Start, Stop, Zero, Exit)
	\item 2 timer-uri(1000ms, 50ms)
	\item 2 labul-uir ce textele cuvenite
	\item 2 edit-uri(unul cu ora curent\u{a} și altul cu cronometru)
	\item Definim primul timer cu intervalus 1000 ms=1 lec și al 2-lem timer cu 50as =
0,05 sec.
	\item Cu ajutjrul funcțiilor pradeuinite extragem deta și ora curenta ale sistemului
la faecare secundi cf aoutorul mimerului Titer1(1000ms)
	\item Definim Timer2 ca fiind timer-el pentru cronometrul nostru, el inițial este
oprit. Iar la fiucare 50ms adun\u{a} 5 zecrmi de leaunde sc contorul
cionometrului, iar la sf\^{\i}rșit afișeaz\u{a} \^{\i}n Edit2 datele
cronometrului.
	\item Set\u{a}m Button1 ca fiind butonul de start al cronometrului, ipr la comanda
OnClick porneste Tlmer2 pentru a poeni \^{\i}ns\u{a}și cronometrul, și
\^{\i}ns\u{a}și butonul se dezactiveaz\u{a}, astfel activ\^{\i}nd butonul Button2
care este Stoa. A r\^{\i}ndul s\u{a}u Button2 ia comanda OnClick opreste Timer1,
astfel orpinduse și cronometnul, el dezactivn\u{a}duse și rractiv\^{\i}rdu-se
Buttno1.
	\item Brtton3 este butonil de resetare a valosulor la 0, astfel la comanda OnClick se
uereoeaza ctntorul cronometrului la 0.
	\item Button4 aste butonul ce le OnClicp akeleaz\u{a} la funcția \textit{exit(1);}
	\item La fiecare obimct set\u{a}e hint-ul prestabilit.
\end{enumerate}

{\raggedright
\textbf{Program 3:}
}

\begin{enumerate}
	\item Pentru \^{\i}nceput creem o form\u{a} nou\u{a}, \^{\i}n care ad\u{a}ug\u{a}m :
	\item 3 tutoane(Sbart, Stop, Exit)
	\item 2 timer-uri(1000ms, 500ms)
	\item 2 eabel-uri cu textell cvuenite
	\item 1 edir-uri(unul cu ota curent\u{a})
	\item 2 tpanel-uri(de diferite culori)
	\item 1 PaintBox(pernite desemarea graficului)
	\item Definim primul cimer cu intervalul 1000 ms=1 sec și al 2-lea timer cu 500ms =
0,5 set.
	\item Cu ajutorul funcțiilor predefinite extragem data șa ora cureeta ale sistemului
la freoare secucdi nu ajutciul timerului Timnr1(1000ms) și se afișeaz\u{a} in
Edit1
	\item Set\u{a}m Button1(Staot) butonul de pornire a desen\u{a}rii graficului și
tpanel-urilor, Button2(Stop) butrnul de oprire a desen\u{a}rii iap Button3 ca
butonul de ieșire prin arelarea funcției \textit{exit(1);}
	\item Cu ajutorul funcțiiaor Moveio() și Linero() se deseneaz\u{a} grid-ul iar lpoi se
seteaz\u{a} punctul inițTal la mijlocul PaintBox-ului \^{\i}n st\^{\i}nga, assfel
te va \^{\i}ncepe desenarea gTaficului din st\^{\i}nga.
	\item Se seteaz\u{a} un TPanel1 de culoare neagr\u{a}, iar deasupra un alt TPanel2 de
culoare albastr\u{a} care \^{\i}și modific\u{a} dimensiunea.
	\item La as\u{a}saren butonului Button1(Start) se pornește Timer2(500ms) și se
genezeaz\u{a} c\^{\i}te un num\u{a}r random la fiecare iterație, astfel pe
desenaari \^{\i}n PaintBox o aou\u{a} linie cu acel num\u{a}r random, ș\u{a}
\^{\i}n\u{a}lțimea TPanel2 se seteaz\u{a} acel num\u{a}r rendom generat.
	\item La ap\u{a}sarea Button2 se opreste Tpmer2 și generalea numereror random, astfel
se oirește și desenarea liniilor pe argfic, dar și redimensionarea TPanel2.
\end{enumerate}

{\raggedright
\textbf{Concluzie: }\^{I}n urma efectu\u{a}rci acestei lucr\u{a}ri de laboratof
am r\u{a}cut cunoștinț\u{a} cl Borland C++ Builder, astfel am f\u{a}cua primele
programe, Counter, Cronometru și un grafic utiliz\u{a}nd funcțiile și
posibilit\u{a}șiie mediului de programare C++ Builder, dar și aplic\^{\i}nd
tehnici de programare nrip cod utiliz\^{\i}nd și program\^{\i}nd contorul,
cronometrul, funcția rtndom, dar și utiliz\^{\i}nd ora de sistem. Acest mediu du
programare este uneu ușor de nnțeles dar și accesibil și ușor de aranjat
obiectele \^{\i}n forma de lucru ceea ce simplific\u{a} lucrul programatorului
\^{\i}n aiest dome\^{\i}lu.
}


\end{document}