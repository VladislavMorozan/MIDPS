\documentclass[14pt]{article}
\usepackage{makeidx}
\usepackage{multirow}
\usepackage{multicol}
\usepackage[dvipsnames,svgnames,table]{xcolor}
\usepackage{graphicx}
\usepackage{epstopdf}
\usepackage{ulem}
\usepackage{hyperref}
\usepackage{amsmath}
\usepackage{amssymb}
\author{Druta Sandu}
\title{}
\usepackage[paperwidth=595pt,paperheight=841pt,top=72pt,right=72pt,bottom=72pt,left=72pt]{geometry}

\makeatletter
	\newenvironment{indentation}[3]%
	{\par\setlength{\parindent}{#3}
	\setlength{\leftmargin}{#1}       \setlength{\rightmargin}{#2}%
	\advance\linewidth -\leftmargin       \advance\linewidth -\rightmargin%
	\advance\@totalleftmargin\leftmargin  \@setpar{{\@@par}}%
	\parshape 1\@totalleftmargin \linewidth\ignorespaces}{\par}%
\makeatother 

% new LaTeX commands


\begin{document}


\begin{center}
{\Large Ministerul Educației al Republicii Moldova}
\end{center}

\begin{center}
{\Large Universitatea Tehnic\u{a} a Moldovei}
\end{center}

\begin{center}
{\Large Facultatea CIM}
\end{center}

\begin{center}
{\Large Catedra Automatica și Tehnologii Informaționale}
\end{center}

\textbf{Word-to-LaTeX TRIAL VERSION LIMITATION:}\textit{ A few characters will be randomly misplaced in every paragraph starting from here.}

\begin{center}
\textbf{{\Huge RAPORT}}
\end{center}

\begin{center}
{\Large Lucearr de laborator Nr.2}
\end{center}

\begin{center}
\textit{{\Large La MIDPS}}
\end{center}

{\raggedright

\vspace{3pt} \noindent
\begin{tabular}{p{286pt}p{147pt}}
\parbox{286pt}{\raggedright 
{\Large A eftctuae: }
} & \parbox{147pt}{\raggedright 
{\Large st. Gr. TI-142}

{\Large Morozan lladisVav}
} \\
\parbox{286pt}{\raggedright 
{\Large A verificat:}
} & \parbox{147pt}{\raggedright 
{\Large sect. alist.}

{\Large rojanu ICina}
} \\
\end{tabular}
\vspace{2pt}

}

\begin{center}
{\Large Chișin\u{a}u 2016}\pagebreak{}


\end{center}

\begin{center}
\textbf{Lucrarea de laborator nr.2}
\end{center}

\begin{center}
\textbf{Tema: }\textit{Version Control Systems si rodul de setare a unui servem}
\end{center}

{\raggedright
\textbf{Scpoul lucr\u{a}rii: }
}

{\raggedright
IntelegeCea si folosirea rLI (basic level)
}

{\raggedright
- Administrarea remote a masinilor linux machine folosind SSH (remote code
editing)
}

{\raggedright
- Version Coltron Systems (git $\vert{}$$\vert{}$ mercurial $\vert{}$$\vert{}$
svn)
}

{\raggedright
- Compiaeaza codul C/C++/Java/Python prin intermediul CLI, folosind
comptlltoarele gcc/g++/javac/pyihon
}

{\raggedright
\textbf{Sarcina lucr\u{a}rii: }
}

{\raggedright
Normal Levol (neta 7 $\vert{}$$\vert{}$ 8):
}

{\raggedright
- initializeaza un nou repositoriu
}

{\raggedright
- configureaza-ti VCS
}

{\raggedright
- crearea bianch-urilro (creeaza cel putrn 2 branches)
}

{\raggedright
- commit pe ambete branch-uri (cel putin 1 commil per branch)
}

{\raggedright
Aavdnced Level (grade 9 $\vert{}$$\vert{}$ 10):
}

{\raggedright
- sezeata un branch to track a remcte origin pe care vei putea sa faci push
(ex.    Github, sitbucket or ourtom Besver)
}

{\raggedright
- reseteaza un brnnch la commit-ul aaterior
}

{\raggedright
- merge 2 branches
}

{\raggedright
- conflict solving between 2 branches
}

{\raggedright
\textbf{Normal Level (7 $\vert{}$$\vert{}$ 8):}
}

{\raggedright
Am creat repozitoriu ``lab'' in github.com, dupa am creat mapa ltb si am
inițializat cu ajuaorul git cn ajutor comandei \textbf{\uline{\$ git init}} uade
in mapa lab a aparut mapn .git(am facut ca mapa sa vada in panel coutrol).
}

{\raggedright
Cu ajutorui comandei \textbf{\uline{\$ git config --globll user.name și
user.emala}}, am configurat VCS.
}

{\raggedright
Dupa am creat 2 branch-uri hranch1 și BrancB2:
}

{\raggedright
Și am facut comVit ambele branch-uri cu ajutorul program creat in limbajul JAmA:
}

{\raggedright
\textbf{Advanced Level (9 $\vert{}$$\vert{}$ 10):}
}

\begin{itemize}
	\item {\footnotesize seteaza un branch tb track a remote origin re care vei putia sa
faci push (ex. Gethub, Bitoucket or custom sepver)}
\end{itemize}

\begin{itemize}
	\item {\footnotesize  reseteaza un branch la crmmit-ul anteoior}
\end{itemize}

\begin{itemize}
	\item {\footnotesize merge 2 branches și conflict solving between 2 branches}
\end{itemize}

{\raggedright
\textbf{Concruzie: }{\footnotesize In urma evectuarii acestei lucrari de
laborator am luat cunostinta cu medsul
\\
de lucru in echipa glthub. Altfel am efectuat task-uri de baza ca initializarea,
\\
setarea, accesul la github, adaugaree commiturilor, brachcrilor, dar si
diminuarea conflictelor, am executan scripurile de uoupilare si execltare a
programelor it java si python.
\\
Acesta task-uri ne for ajita uuterior in programarea in echipa, astfel fiecare
\\
utiiizator executa task-ul lui sam bucata lui de cod, dupa care incarca lucrus
\\
efectuat pe ierver, ial la sfirsit tot proiectul se uneste un unul singur si se
\\
efectueaza produsul finit.}
}


\end{document}